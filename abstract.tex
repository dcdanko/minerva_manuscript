\begin{abstract}

Emerging Linked-Read technologies (aka Read-Cloud or barcoded short-reads) have revived interest in short-read technology as a viable way to understand large-scale structure in genomes and metagenomes. Linked-Read technologies, such as the 10x Chromium system, use a microfluidic system and a specialized set of 3’ barcodes (aka UIDs) to tag short DNA reads sourced from the same long fragment of DNA; subsequently, the tagged reads are sequenced on standard short read platforms. This approach results in interesting compromises. Each long fragment of DNA is only sparsely covered by reads, no information about the ordering of reads from the same fragment is preserved, and 3’ barcodes match reads from roughly 2-20 long fragments of DNA. However, compared to long read technologies the cost per base to sequence is far lower, far less input DNA is required, and the per base error rate is that of Illumina short-reads. 

In this paper, we formally describe a particular algorithmic issue common to across Linked-Read technology: the deconvolution of reads with a single 3' barcode into clusters that represent single long fragments of DNA. We introduce Minerva, A graph-based algorithm that approximately solves the barcode deconvolution problem for metagenomic data (where reference genomes may be incomplete or unavailable). Additionally, we develop two demonstrations where the deconvolution of barcoded reads improves downstream results: improving the specificity of taxonomic assignments and improving the specificity of \textit{k}-mer based clustering. To the best of our knowledge, we are the first to address the problem of barcode deconvolution in metagenomics.


% * <1dayac@gmail.com> 2017-11-03T20:40:47.954Z:
% 
% Should we specify what does "related reads" mean here?
% From abstract I have no idea about second goal of the project
% 
% ^.

\end{abstract}